\documentclass{article}
\usepackage{graphicx} % Required for inserting images
\usepackage[czech]{babel}

\title{Vacation Planner}
\author{
    T. Daniel (xdanie14),
    M. Slezák (xsleza26),
    J. A. Štigler (xstigl00)
}
% \date{Date}

\begin{document}

\maketitle

\newpage

\section{Téma}
% ad 1) – Každý člen týmu vypracuje unikátní návrh tématu projektu.
%   - Inspirujte se již existujícími aplikacemi, svými vlastními potřebami a
%     potřebami lidí ve vašem okolí.
%   - Ideální téma vychází z reálných potřeb uživatelů, kteří jsou nespokojeni
%     s dostupnými řešeními. Tématem pak může být vylepšení nevyhovující
%     existující aplikace nebo vytvoření aplikace úplně nové.
%   - Zpráva o návrhu: popište Vámi navržené téma a uveďte zdůvodnění tématu ve
%     formě naplnění konkrétních uživatelských potřeb.

\subsection{Vacation Planner}
Při plánování dovolené je často potřeba naplánovat trasu a~zjistit kolik co
stojí, otevírací dobu a~tak dále. Při takovém plánování však člověk trasu
plánuje v~mapách, dodatečné informace hledá na internetu. Mnohdy si vše zapíše
na papír a~poté vyznat se v~tom všem je velmi obtížné.

Mnohem jednodušším způsobem by bylo mít všechny informace na jednom místě.
Návrhem je tedy vytvořit aplikaci, která bude obsahovat všechny důležité
informace o~dovolené. Uživatel si bude moct vytvořit plán dovolené, který
pak bude obsahovat jednotlivé výlety, které chce podniknout. Výlet zase bude
uchovávat zastávky, které si naplánoval, a~všechny důležíté informace jako je
cena a~otevírací doba.

\subsection{Game Launcher}
V dnešní době existuje mnoho počítačových her a s nimi i mnoho obchodů.
Každý obchod a i někteří vydavatelé her mají vlastní aplikaci, ve které má
uživatel seznam her, co vlastní, a může zde hry stahovat, spouštět a dívat se
na různé statistiky (například takzvané achievementy).

Je normální, že jeden uživatel používá několik takových launcherů. To je ale
nepříjemné, protože když si chce zapnout nějakou hru, tak musí vědět, ve kterém
launcheru ji má. Návrh je vytvořit univerzální launcher, který dokáže
automaticky rozpoznat a načíst hry z různých používaných launcherů, a tak vše
sjednotit pod jednu aplikaci. Takové aplikace existují, ale mají své nevýhody.
Například fungují jen na jednom operačním systému nebo je potřeba každou hru
přidávat manuálně.

\subsection{TODO}
TODO: Tom

\subsection{Vybrané téma}
Vybrali jsme téma Vacation Planner. Shodli jsme se že je to nejlepší téma kde
se dá vyhrát s frontendem aniž by jsme potřebovali moc backendu.

\section{Průzkum uživatelsých potřeb}

\subsection{Dotazník}
% ad 3) – Každý člen provede analýzu uživatelských potřeb a klíčových problémů.
%   - Připravte si dotazník.
%   - Proveďte průzkum mezi potenciálními uživateli pomocí rozhovorů a Vámi
%     navrženého dotazníku.
%   - Na základě průzkumu proveďte analýzu uživatelských potřeb a klíčových
%     problémů.
%   - Počet dotazovaných uživatelů musí být minimálně 2. Uživatelé musí být
%     relevantní, tzn. reální uživatelé dané aplikace (úředník, číšník, učitel,
%     řidič, hudebník, kuchař apod.).
%   - Zpráva o návrhu: uveďte dotazník ve formě jednotlivých otázek, poznatky
%     získané z odpovědí, analýzu těchto odpovědí s důrazem na potřeby
%     uživatelů a klíčových problémů.

TODO

\subsection{Existující aplikace}
% ad 4) – Každý člen vyhledá jednu existující aplikaci, která řeší zvolené téma
%         a na základě provedené analýzy uživatelských potřeb popíše přednosti
%         a nedostatky dané aplikace.
%   - Zjistěte přednosti a nedostatky vybrané aplikace s využitím informací z
%     průzkumu uživatelských potřeb (bod 3).
%   - Na základě toho si ujasněte, jakou funkcionalitu by měla mít vaše budoucí
%     aplikace a jaké by se naopak měla vyvarovat.
%   - Zpráva o návrhu: krátce popište Vámi nalezenou existující aplikaci, její
%     přednosti a nedostatky; jak se vaše budoucí aplikace bude inspirovat
%     přednostmi a jak bude řešit nedostatky

TODO

\subsubsection{Mapy.cz}
TODO

\subsection{Uživatelské potřeby, klíčové vlastnosti}
% ad 5) – Všichni členové se shodnou na uživatelských potřebách a klíčových
%         problémech.
%   - Na základě uživatelských průzkumů, analýz uživatelských potřeb, klíčových
%     problémů a výhod/nevýhod existujících aplikací se všichni členové týmu
%     shodnou na klíčových vlastnostech budoucí aplikace, tyto vlastnosti musí
%     odrážet konkrétní potřeby uživatelů, které jednotliví členové zjistili v
%     rámci bodu 3 a 4.
%   - Zpráva o návrhu: uveďte klíčové potřeby uživatelů, uživatelské procesy
%     (jak uživatelé postupují) a klíčové vlastnosti vaší budoucí aplikace,
%     které budou tyto potřeby řešit.

TODO

\section{Návrh}

\subsection{Rozdělení práce}
% ad 6) – Všichni členové se shodnou na rozdělení práce mezi jednotlivé členy.
%   - Rozdělení práce lze provést dvěma způsoby:
%       1. Každý člen pracuje na unikátním/vlastním řešení aplikace, výsledkem
%          jsou 3 aplikace, jejichž vlastnosti GUI lze na závěr práce porovnat.
%       2. Každý člen pracuje na části aplikace, výsledkem je jedna aplikace.
%   - Každý člen týmu musí v projektu sám autorsky realizovat/implementovat
%     nějakou část aplikace s GUI. Práce v týmu slouží ke snížení režie s
%     realizací backendu, ke vzájemné podpoře a sdílení implementačních
%     zkušeností a především k získání zkušenosti z práce v týmu.
%   - Zpráva o návrhu: jasně uveďte rozdělení práce (kdo bude na čem pracovat)
%     a pro jaký způsob rozdělení práce jste se rozhodli (1 nebo 2).

TODO

\subsection{Návrh}
% ad 7) – Každý člen navrhne informační strukturu a GUI své části a vytvoří maketu
%         s ohledem na uživatelské potřeby.
%   - Navrhněte rozložení informace a interakce do oddílů
%     (stránek/obrazovek/sekcí atd.) a rozložení GUI prvků v jednotlivých
%     oddílech. Návrh informační struktury vysvětlete, tedy popište důvody,
%     proč je GUI rozděleno právě takto, jaké jsou mezi GUI prvky logické vazby
%     atd.
%   - Vytvořte maketu GUI pomocí programu Figma. Ne kompletní prototyp (ne
%     všechny funkce 1:1 s výslednou aplikací), ale maketu (viz přednášky).
%   - Návrh musí reflektovat vlastnosti aplikace, na kterých se členové shodli
%     v rámci bodu 5.
%   - Zpráva návrhu: krátce popište jak váš návrh řeší potřeby uživatelů
%     (odkazujte se na bod 5), maketu reprezentujte pomocí obrázků, diagramů
%     nebo snímků obrazovky a vhodného krátkého popisu.

TODO

\end{document}
